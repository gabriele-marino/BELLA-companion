\begin{abstract}

    Phylodynamic modeling quantifies the parameters that govern the growth of phylogenetic trees. Current extensions using generalized linear models (GLMs) allow the integration of external covariates, such as host traits or mobility data, but are limited to linear and additive effects. To overcome this limitation, we introduce a Bayesian neural network (BNN) approach within a Markov chain Monte Carlo (MCMC) phylodynamic framework. Rather than sampling parameters directly, the method samples BNN weights, enabling the network to learn functional mappings from predictors to rates of speciation, extinction, migration, or transmission. Simulations show that this framework accurately recovers predictor–parameter relationships, avoids overfitting, and performs robustly across a range of dynamics. Using explainable AI techniques, we show that our framework is able to correctly identify the predictors with the greatest influence, providing both interpretability and precision. We further validate the method on empirical datasets, estimating COVID-19 migration rates during its early spread in Europe and inferring speciation and extinction dynamics in platyrrhine phylogenies, proving that our unsupervised BNN framework provides a flexible and powerful tool for modeling complex epidemiological and macroevolutionary processes.

\end{abstract}
