\section{Discussion}

In this study, we have introduced a novel Bayesian neural network (BNN) framework for phylodynamic inference that effectively integrates non-genetic covariates to model complex evolutionary dynamics. Our model uses multilayer perceptrons (MLPs) within a Markov chain Monte Carlo (MCMC) setting to unsupervisedly learn functional mappings from predictors to key parameters such as migration, speciation, and extinction rates, overcoming the limitations of traditional generalized linear models (GLMs) that are restricted to linear and additive effects. We benchmarked our approach through extensive simulations, including four epidemiological and four macroevolutionary scenarios, with a diverse range of predictor-parameter relationships, evaluating the performance of three different MLP architectures against predictor-agnostic and GLM models in terms of estimation accuracy and precision. We analysed our results by grouping scenarios that shared similar dynamics (constant, linear, nonlinear) and found that our BNN framework consistently matches or exceeds the performance of state-of-the-art phylodynamic models across all scenarios.

In scenarios characterized by constant predictor-parameter relationships (Epi-1 and FBD-1), our smallest MLP architecture outperformed all other models, achieving the lowest mean absolute error (MAE), highest coverage, and narrowest credible intervals. Its performance even exceeded that of the GLM—despite the latter being well suited to linear settings and the MLP having a larger parameter space. Larger MLP architectures performed comparably well, including the largest network tested, demonstrating a notable robustness to overfitting in simple settings, even when the network size was suboptimal. A slight decrease in precision was observed as model complexity increased. In contrast, the predictor-agnostic model performed substantially worse in terms of MAE and credible interval width, while maintaining acceptable coverage across all scenarios. This pattern was consistently observed across all simulation studies and is largely attributable to the unconstrained parameterization of the predictor-agnostic model, which hinders accurate recovery of constant underlying processes and produces unstable estimates with systematically inflated errors. The resulting uncertainty was particularly pronounced in early time bins, where limited data make it difficult for the model to infer stable parameters, though this issue diminishes as phylogenetic trees accumulate more tips and branching events over time. When tested on linear dynamics (Epi-2, FBD-2, FBD-4), we observed similar trends: the MLPs performed on par with the GLM in Epi-2, and even outperformed it in FBD-2 and FBD-4, with the intermediate and larger architectures achieving the best results. These findings, together with those from the constant scenarios, are particularly noteworthy given that GLMs are specifically designed to model linear relationships. Yet, our Bayesian neural network (BNN) framework demonstrated comparable or superior performance, while flexibly adapting to the data without prior assumptions about functional form.

Under nonlinear dynamics in scenarios Epi-3 and Epi-4, the intermediate and larger MLP architectures clearly outperformed the GLM in terms of MAE and coverage, as the GLM was unable to capture nonlinear relationships due to its inherent linearity assumptions. The smallest MLP architecture showed somewhat reduced performance relative to its larger counterparts, likely due to its limited capacity to model complex nonlinear functions, though it still surpassed the GLM.\ Both intermediate and larger MLP models also outperformed the predictor-agnostic model, producing more precise estimates with narrower credible intervals and lower MAE, particularly in Epi-4. The final nonlinear scenario, FBD-3, presented a more challenging case, involving a nonlinearly decreasing speciation rate coupled with a spiking extinction rate. As a continuous model, the Bayesian MLP is not ideally suited to capturing such sharp, transient events: while it successfully identified the overall trend, it tended to smooth the spike, leading to reduced coverage. Nonetheless, increasing the model size improved performance, and all MLP architectures outperformed both the GLM and the predictor-agnostic models in terms of MAE.\ Overall, these results illustrate that Bayesian MLPs provide accurate and credible reconstructions of nonlinear dynamics, though they are less effective at modeling abrupt, spiking events. In such cases, high coverage from more flexible models like BDSky may reflect uncertainty rather than precise inference, whereas the MLP framework offers a more balanced trade-off between predictive accuracy and credible uncertainty.

When selecting an MLP architecture, the complexity of the underlying parameter-predictor relationships should be taken into account. In our experiments, smaller architectures generally performed better in scenarios with simpler, more constant relationships, while larger networks were more effective at capturing nonlinear, continuously changing, or abruptly changing dynamics. These differences in predictive performance, however, were not decisive. The number of parameters also affects computational speed: MLP models are typically slower than the predictor-agnostic and GLM frameworks, and larger networks require more time to converge. Nonetheless, all tested MLP configurations were computationally feasible, and runtime differences were not prohibitive. Depending on the scenario, smaller MLPs tend to converge faster in simple settings, whereas larger architectures may be preferable when multiple predictors or more complex dynamics are involved. This underscores the importance of balancing model complexity, predictive performance, and computational efficiency when designing Bayesian neural networks.

Interpreting complex models is critical in evolutionary and phylodynamic inference, as understanding which predictors drive parameter estimates provides both biological insight and confidence in model outputs. Our analyses showed that the Bayesian MLP models accurately recovered relationships between relevant predictors and target parameters while remaining largely insensitive to irrelevant or random inputs. For instance, in scenario Epi-4, the model captured the expected sigmoidal relationship for migration rates, and in FBD-4, it effectively identified temporal trends and the effects of relevant traits on speciation and extinction rates. Importantly, we demonstrated that these models are interpretable using partial dependence plots (PDPs) and Shapley values (SHAP), which allow quantification of each predictor's influence on model outputs. The ability to distinguish relevant from irrelevant features is particularly valuable in high-dimensional or noisy settings, where spurious predictors could otherwise bias inference. By correctly attributing effects to biologically meaningful variables while remaining largely insensitive to random or irrelevant inputs, Bayesian MLPs combine predictive accuracy with transparency. This capability supports robust inference in complex scenarios, including nonlinear dynamics or multi-class evolutionary processes, where traditional parametric models may struggle to capture subtle or interacting effects. Overall, these findings highlight that Bayesian MLPs provide a flexible and interpretable framework, capable of integrating complex predictor information while maintaining clarity about which factors truly drive predictions.

Beyond phylodynamic applications, the proposed Bayesian MLP framework holds promise for a wide range of evolutionary inference tasks. The same model structure could be applied to non-phylodynamic settings such as estimating phylogenetic substitution rates, molecular clock parameters, or other evolutionary processes where parameter dynamics depend on external predictors. Its flexibility in capturing both linear and nonlinear relationships without strong prior assumptions makes it a compelling alternative to traditional parametric models, potentially enabling more accurate and data-driven reconstructions of complex evolutionary patterns.
